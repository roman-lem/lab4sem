\input{preambule_article.tex}


\begin{document}

\input{title.tex}
\section{Аннотация}
\textbf{Цель работы:} изучить модели зрительных труб (астрономической
трубы Кеплера и земной трубы Галилея) и микроскопа, определить
их увеличения.

\textbf{Приборы и материалы:} оптическая скамья, набор линз, экран,
осветитель со шкалой, зрительная труба, диафрагма, линейка.

\section{Теоретические сведения}
	
	В работе предлагается измерить фокусные расстояния линз, смоделировать трубу Кеплера, трубу Галилея, микроскоп и определить их увеличения. Предметом служит миллиметровая сетка, нанесённая на матовое стекло осветителя.
	
	При юстировке любых оптических приборов важно правильно центрировать входящие в систему линзы. Проходя через плохо отцентрированную систему линз, лучи света отклоняются в сторону и могут вообще не доходить до глаза наблюдателя.
	
	Для построения телескопических систем необходим удаленный объект. Эту роль выполняет коллиматор настроенный на бесконечность.
	
	Фокусные расстояния положительных линз проще всего найти с помощью вспомогательной зрительной трубы, установленной на бесконечность. 
	
	\subsection{Определение фокусных расстояний тонких линз с помощью
		зрительной трубы}
	\begin{figure}[H]
		\centering
		\includegraphics[scale=0.7]{focus}
	\end{figure}

	Определение фокусного расстояния линзы происходит с помощью зрительной трубы настроенной на бесконечность. Так как лучи от сетки, расположенной в фокусе после прохождения положительной линзы идут параллельно. А чтобы достичь того же эффекта от отрицательной линзы, нужно разместить перед ней собирающую.
	
	\begin{figure}[H]
		\centering
		\includegraphics[scale=0.5]{focus2}
	\end{figure}

	\subsection{Телескоп Кеплера}
	
	\begin{figure}[H]
		\centering
		\includegraphics[scale=0.4]{kepler}
	\end{figure}

	Увеличение этой модели телескопа рассчитывается по формуле
	\[
		\Gamma_K = -\frac{f_1}{f_2}
	\]
	Или для телесных углов
	\[
	\Gamma_K = -\frac{h_2}{h_1}
	\]
	Или для диаметра диафрагмы
	\[
	\Gamma_K = -\frac{D_1}{D_2}
	\]
	
	\subsection{Труба Галилея}
		
	Данная оптическая система отличается от трубы Кеплера только тем, что в качестве окуляра берется рассеивающая линза.
	
	\begin{figure}[H]
		\centering
		\includegraphics[scale=0.3]{gal}
	\end{figure}

	\subsection{Модель микроскопа}
	
	Оптическая схема микроскопа выглядит следующим образом
	\begin{figure}[H]
		\centering
		\includegraphics[scale=0.5]{micro}
	\end{figure}

	Его увеличение можно найти так
	\[
		\Gamma_M = \frac{h_2L}{h_1f}
	\]
	где $f$ - фокусное расстояние коллиматорной линзы, используемой для измерения $h_1$, А $L = 25$ см - расстояние наилучшего зрения.
	
	\section{Обработка результатов}
	После центрировки оптической системы найдем фокусные расстояния собирающих линз.
	\subsection{Определение фокусных расстояний тонких линз}
	Отберем из набора собирающие линзы и с помощью схемы 2.1 определим их фокусное расстояние. Погрешность обусловлена невозможностью определить точное положение линзы внутри оправы.
	
	\begin{table}[H]
		\centering
		\begin{tabular}{|l|l|l|l|l|}
			\hline
			$N$ & $f_1^+$, см  & $f_2^+$, см  & $f^+$, см & $\sigma_{f}$, см \\ \hline
			1   & $7,6\pm0,2$  & $7,5\pm0,2$  & $7,55$    & 0,14             \\ \hline
			2   & $10,5\pm0,2$ & $10,6\pm0,2$ & $10,55$   & 0,14             \\ \hline
			3   & $19,2\pm0,3$ & $18,9\pm0,3$ & $19,1$    & 0,2              \\ \hline
			4   & $28,2\pm0,3$ & $28,0\pm0,3$ & $28,1$    & 0,2              \\ \hline
		\end{tabular}
	\end{table} 
	Погрешность определим по формуле косвенных измерений
	\[
		\sigma_f = \sqrt{\left(\frac{\sigma_+}{2}\right)^2+\left(\frac{\sigma_-}{2}\right)^2}
	\]
	
	Чтобы определить фокусное расстояние рассеивающей линзы поместим собирающую после источника и найдем расстояние до изображения $a_0 = 34,4\pm0,2$ см. Тогда фокусное расстояние собирающей линзы:
	\begin{table}[H]
		\centering
		\begin{tabular}{|l|l|l|l|l|}
			\hline
			$N$ & $l_1^-$, см  & $l_2^-$, см  & $l^-$, см    & $f^-$, см   \\ \hline
			5   & $27,9\pm0,3$ & $27,7\pm0,3$ & $27,8\pm0,2$ & $-6,6\pm0,1$ \\ \hline
		\end{tabular}
	\end{table}

	\subsection{Телескоп Кеплера}
	В качестве коллиматора возьмем линзу №3 ($f=19,1$ см), сам телескоп соберем из линз 4 и 2 ($f_1=28,1$ см, $f_2=10,55$ см).
	
	Запишем данные для определения увеличения:
	\begin{table}[H]
		\centering
		\begin{tabular}{|l|l|l|l|}
			\hline
			$h_1$, дел   & $h_2$, дел & $D_1$, мм & $D_2$, мм  \\ \hline
			$10,0\pm0,5$ & $28\pm1$   & $36\pm1$  & $13\pm0,5$ \\ \hline
		\end{tabular}
	\end{table}
	и определим увеличение разными способами
	\[
		\Gamma_K=-\frac{f_1}{f_2} = -2,66\pm0,04
	\]
	\[
		\Gamma_K=-\frac{h_2}{h_1} = -2,8\pm0,17
	\]
	\[
		\Gamma_K=-\frac{D_1}{D_2} = -2,7\pm 0,13
	\]
	Погрешность формул вида $x/y$ рассчитывается по формуле
	\[
		\sigma = \sqrt{\left(\frac{\sigma_x}{y}\right)^2 + \left(\frac{x\sigma_y}{y^2}\right)^2}
	\]
	Видно, что результаты сходятся в пределах погрешности.
	
	\subsection{Труба Галилея}
	В качестве объектива и коллиматора оставим те же линзы, что и в предыдущем опыте. А в качестве окуляра возьмем рассеивающую линзу ($f = -6,6$ см).
	
	Запишем результаты измерений.
	\begin{table}[H]
		\centering
		\begin{tabular}{|l|l|}
			\hline
			$h_1$, дел   & $h_2$, дел \\ \hline
			$10,0\pm0,5$ & $42\pm2$   \\ \hline
		\end{tabular}
	\end{table}

	\[
		\Gamma_G = -\frac{f_1}{f_2} = 4,25\pm0,07
	\]
	\[
		\Gamma_G = \frac{h_2}{h_1} = 4,2\pm 0,2
	\]
	Результаты сходятся в пределах погрешности.
	
	\subsection{Модель микроскопа}
	Для этой модели отберем из набора линзы ($f_1 = 7,55$ см) и ($f_2 = 10,55$ см). Желаемое увеличение $\Gamma_M = 5$. Согласно формуле $\Gamma_M = \dfrac{\Delta L}{f_1f_2}$ и $\Delta = l_{12} - f_1 - f_2$. Найдем длину тубуса $l_{12} = 34$ см.
	
	Измерим величину изображения миллиметрового деления предметной шкалы. $h_2 \hm= 37\pm1$ дел. И найдем увеличение микроскопа по формуле
	\[
		\Gamma_M = -\frac{h_2L}{h_1f} = 4,8\pm2
	\]
	\[
		\sigma_M = \sqrt{\left(\frac{\sigma_{h_2}L}{h_1f}\right)^2 + \left(\frac{\sigma_{h_1}h_2L}{h_1^2f}\right)^2+ \left(\frac{\sigma_{f}h_2L}{h_1f^2}\right)^2}
	\]
	Хотя, формально желаемый результат попадает в погрешность, отличие все же значительное. Это может быть связано со сложностью установления длины тубуса согласно рассчитанному значению.
	\section{Вывод}
	В этой работе мы исследовали модели зрительных труб и микроскопа, а так же определили их увеличения различными способами.
	
	
	
	
	
	
	
	
	
	
	
	
	
	
	
	
	
	
	

\end{document}