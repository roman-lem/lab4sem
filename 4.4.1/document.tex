\input{preambule_article.tex}


\begin{document}

\input{title.tex}

\section{Аннотация}
\textbf{Цель работы:} знакомство с работой и настройкой гониометра Г5, определение спектральных характеристик амплитудной решетки.

\textbf{В работе используются:}  гониометр, дифракционная решетка, ртутная лампа.
\section{Теоретические сведения}
\noindent Основное соотношение приближенной теории дифракционной решётки:
\begin{equation}
d\sin \varphi_m = m\lambda.
\end{equation}
Угловая дисперсия $D$ характеризует угловое расстояние между близкими спектральными линиями:
\begin{equation}
D = \frac{d\varphi}{d\lambda} = \frac{m}{d \cos \varphi}=\frac{m}{\sqrt{d^{2}-m^{2} \lambda^{2}}}.
\end{equation}

\section{Экспериментальная установка}
При работе с дифракционной решёткой основной задачей является точное измерение углов, при которых наблюдаются главные максимумы для различных длин волн. В нашей работе для измерения углов используется гониометр Г5. Принципиальная схема экспериментальной установки приведена на рис. \ref{inst}.
\begin{figure}[H]
	\centering
	\includegraphics[scale=1.5]{inst}
	\caption{Схема установки.}
\end{figure}

\section{Обработка результатов}

	Запишем угловые координаты линий $\pm1$ порядка. Погрешность измерений составляет $5''$ или $2\cdot 10^{-5}$ рад. Поскольку невозможно точно настроиться на центр полосы.
	
	\begin{table}[H]
		\centering
		\begin{tabular}{|l|l|l|l|l|l|}
			\hline
			Цвет       & $\lambda$, нм & $\varphi_{+1}$, рад & $\varphi_{-1}$, рад & $\sin{\varphi_{+1}}$ & $\sin{\varphi_{-1}}$ \\ \hline
			красный    & -             & 0,31735             & -0,30785            & 0,31205              & -0,30301             \\ \hline
			красный    & -             & 0,31444             & -0,30495            & 0,30929              & -0,30024             \\ \hline
			желтый     & 579,1         & 0,29971             & -0,29037            & 0,29524              & -0,28631             \\ \hline
			желтый     & 577,0         & 0,29957             & -0,29029            & 0,29511              & -0,28623             \\ \hline
			зеленый    & 546,1         & 0,28217             & -0,27350            & 0,27844              & -0,27010             \\ \hline
			голубой    & 491,6         & 0,25310             & -0,24618            & 0,25041              & -0,24370             \\ \hline
			фиолетовый & 404,7         & 0,22399             & -0,21922            & 0,22212              & -0,21747             \\ \hline
		\end{tabular}
	\end{table}

	Погрешность синуса найдем следующим образом:
	\[
		\sigma_{\sin} = \sqrt{\left(\frac{\partial \sin\varphi}{\partial \varphi}\cdot\sigma_\varphi \right)^2} = |\cos\varphi\cdot\sigma_\varphi| \approx 2\cdot 10^{-5}
	\]
	
	Для линий спектра с известной длиной волны построим график зависимости $\sin{\varphi}$ от $\lambda$.
	
	\begin{figure}[H]
		\centering
		\includegraphics[scale=0.7]{gr1}
	\end{figure}

	Коэффициент наклона $k = (41\pm2)\cdot 10^{-5}$ нм$^{-1}$
	\[
		d= \frac{1}{k} = 2,43 \pm 0,11 \text{ мкм}
	\]
	\[
		\sigma_d = \frac{\sigma_k}{k^2}
	\]
	Значение шага, написанное на амплитудной решетке: $2$ мкм.
	
	Рассчитаем угловую дисперсию для спектров разного порядка. Для желтого дублета $\delta\lambda = 2,1$ нм.
	
	\begin{table}[H]
		\centering
		\begin{tabular}{|l|l|l|l|l|}
			\hline
			$m$ & $\varphi_{+1}$, рад & $\varphi_{-1}$, рад & $\delta\varphi$, рад & $D$, 1/нм \\ \hline
			-1  & -0,29009            & -0,29037            & 0,00028              & 0,00013   \\ \hline
			-2  & -0,59780            & -0,59949            & 0,00169              & 0,00080   \\ \hline
			+1  & 0,29957             & 0,29980             & 0,00023              & 0,00010   \\ \hline
			+2  & 0,64297             & 0,64586             & 0,00289              & 0,00137   \\ \hline
		\end{tabular}
	\end{table}
	\[
		\sigma_D = \frac{\sigma_{\delta\varphi}}{\delta\lambda} = 1\cdot 10^{-5} \text{ 1/нм}
	\]
	График зависимости угловой дисперсии от порядка спектра
	\begin{figure}[H]
		\centering
		\includegraphics[scale=0.7]{gr2}
	\end{figure}

	Формула (2) дает значения для 1 порядка $D_1 = 5 \cdot 10^{-4}$ 1/нм, и для 2 порядка $D_2 \hm= 12 \cdot 10^{-4}$ 1/нм. Полученные значения совпадают по порядку.
	
	Оценим разрешимый спектральный интервал $\delta\lambda$
	\[
		\delta\lambda = \frac{\Delta\varphi}{D}
	\]
	\begin{table}[H]
		\centering
		\begin{tabular}{|l|l|l|l|l|}
			\hline
			$m$ & $\varphi_{+1}$, рад & $\varphi_{-1}$, рад & $\delta\varphi$, $10^{-5}$ рад & $\delta\lambda$, нм \\ \hline
			-1  & -0,29037            & -0,29038            & 1                              & 0,075               \\ \hline
			-2  & -0,59798            & -0,59796            & 2                              & 0,024               \\ \hline
			+1  & 0,30005             & 0,30004             & 1                              & 0,091               \\ \hline
			+2  & 0,64619             & 0,64616             & 3                              & 0,021               \\ \hline
		\end{tabular}
	\end{table}

	Усредняя получим
	\[
		\delta\lambda = 0,05\pm0,03 \text{ нм}
	\]
	Тогда, разрешающая способность 
	\[
		R = \frac{\lambda}{\delta\lambda} = 11000\pm7000
	\]
	
	Найдем число эффективно работающих штрихов $N = R/m$
	\begin{table}[H]
		\centering
		\begin{tabular}{|l|l|}
			\hline
			$m$ & $N$, $10^3$ \\ \hline
			1   & $11\pm7$    \\ \hline
			2   & $5\pm3$     \\ \hline
		\end{tabular}
	\end{table}

	Тогда эффективный размер решетки $l = Nd$:
	\begin{table}[H]
		\centering
		\begin{tabular}{|l|l|}
			\hline
			$m$ & $l$, мм   \\ \hline
			1   & $22\pm14$ \\ \hline
			2   & $11\pm7$  \\ \hline
		\end{tabular}
	\end{table}

	Можно оценить, что желтая линия наложится на фиолетовую при $m=6$ для желтого и $m =8$ для фиолетового. Поскольку
	\[
		0,29971\cdot6-0,22399\cdot8 = 0,00634
	\]
	
	\section{Вывод}
	В данной работе мы познакомились с устройством гониометра, а так же определили спектральные характеристики амплитудной решетки.
	
	

\end{document}
