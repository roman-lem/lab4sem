\input{preambule_article.tex}


\begin{document}

\input{title.tex}

\section{Аннотация}

\textbf{Цель работы: } знакомство с устройством и работой газового лазера непрерывного действия, со спектральными характеристиками лазерного излучения, а также с устройством и принципом действия сканирующего интерферометра Фабри—Перо.

\textbf{В работе используются: } Не–Nе-лазер с блоком питания; сканирующий интерферометр Фабри—Перо; поляроид; пластинка $\lambda/4$; линза; фотодиод; электронный осциллограф.

\section{Теоретические сведения}

\begin{figure}[H]
	\centering
	\includegraphics[scale = 0.7]{laser_scheme.png}
	\caption{Схема лазера}
\end{figure}

В лазере генерируются моды для которых на длине лазера укладывается целое число полуволн: $2L = m\lambda$, откуда межмодовое расстояние:
\begin{equation}
    \label{equ:betweenMode}
	\nu_{m+1} - \nu_{m} = \frac{c}{2L}
\end{equation}

\subsection{Сканирующий интерферометр}

\begin{figure}[H]
	\centering
	\includegraphics[scale = 0.7]{interf.png}
	\caption{Интерферометр Фабри-Перо}
\end{figure}

Интерферометр Фабри-Перо представляет собой 2 зеркала, одно из которых расположено на пьезоэлементе, что позволяет изменять расстояние между ними на величину порядка длины волны. По аналогии с лазером, при выполнении условия $2l = m\lambda$, возникает резонанс.
Если на интерферометр падает излучение с различными длинами волн, то одновременно может возникнуть несколько резонансов. Собственные моды интерферометра отличаются по частоте на величину
\begin{equation}
	\label{equ:dispObl}
	\Delta\nu = \frac{c}{2l}
\end{equation}
которая называется дисперсионной областью. 

Разрешающая способность $R$ спектрального прибора определяется соотношением:
\begin{equation*}
	R = \frac{\nu}{\delta\nu}
\end{equation*}

Разрешающую способность интерферометра Фабри-Перо можно рассчитать по формуле
\begin{equation}
\label{equ:prob}
	R = \frac{2\pi l}{\lambda (1-r)}
\end{equation}

\section{Экспериментальная установка}

\begin{figure}[H]
	\centering
	\includegraphics[scale = 0.7]{stand.png}
	\caption{Схема экспериментальной установки}
\end{figure}

Луч, вышедший из лазера, проходит через поляризационную развязку, дабы не допустить попадания отраженного света в лазер. Далее, фокусируется линзой, проходит через резонатор и попадает на фотодиод, подключенный к осциллографу. Таким образом мы можем наблюдать периодическое изменение интенсивности.

\section{Обработка результатов}

Запишем данные установки:
\begin{table}[H]
	\centering
	\begin{tabular}{|l|l|l|}
		\hline
		$L$, м & $\lambda$, нм & $l$, м \\ \hline
		0,65   & 632,8         & 0,09   \\ \hline
	\end{tabular}
\end{table}
Отсюда, по формуле (\ref{equ:betweenMode}) найдем межмодовое расстояние $\Delta\nu = 230,7$ МГц.
Преобразуем его в единицы $\Delta\lambda$ следующим образом:
\begin{equation*}
	\Delta\nu = \frac{c}{\lambda_{m+1}} - \frac{c}{\lambda_{m}} \approx \frac{c\Delta\lambda}{\lambda^2} \Rightarrow \Delta\lambda = \frac{\lambda^2\Delta\nu}{c} = 3\cdot 10^{-4} \text{ нм}
\end{equation*}
Откуда, ширина спектра генерации лазера $\Delta\lambda(Ne) = 1.8\cdot10^{-3}$ нм.
Полагая, что уширение спектра обусловлено эффектом Доплера, найдем среднюю скорость движения атомов в направлении оптической оси.
\begin{equation*}
	\frac{V_x}{c} \approx \frac{\Delta\lambda(Ne)}{\lambda} \Rightarrow V_x = c\frac{\Delta\lambda(Ne)}{\lambda} = 853,3 \text{ м/с}
\end{equation*}
Найдем газокинетическую температуру $T$ в разряде
\begin{equation*}
	\frac{m(Ne)V_x^2}{2} = \frac{k_BT}{2} \Rightarrow T = \frac{m(Ne)V_x^2}{k_B} = 1768 \text{ К}
\end{equation*}

Согласно формуле (\ref{equ:dispObl}) найдем дисперсионную область
\begin{equation*}
	\Delta\lambda_{si} = \frac{\lambda^2\Delta\nu}{c} = \frac{\lambda^2}{2l} = 2\cdot10^{-3} \text{ нм} \sim \Delta\lambda(Ne)
\end{equation*}

Сравнив ширину отдельной моды с межмодовым расстоянием найдем разрешение $\delta\nu$
\begin{table}[H]
	\centering
	\begin{tabular}{|l|l|}
		\hline
		$\delta\nu$, дел & $\Delta\nu$, дел \\ \hline
		$0,2\pm 0,1$     & $0,6\pm 0,1$     \\ \hline
	\end{tabular}
\end{table}
Цена деления: 1 дел = $384,5$ МГц
Разрешение: $\delta\nu = 70 \pm 30$ МГц

Посчитаем разрешающую способность $R$
\begin{equation*}
	R = \frac{\nu}{\delta\nu} =\frac{c}{\lambda\delta\nu} = (6 \pm 2)\cdot10^{6}
\end{equation*}
Отсюда, по формуле (\ref{equ:prob}) найдем коэффициент отражения
\begin{equation*}
	r= 1-\frac{2\pi l}{\lambda R} = 0,85\pm 0,04
\end{equation*}

\begin{figure}[H]
	\centering
	\includegraphics[scale = 0.7]{spectr.jpg}
	\caption{Картина спектра}
\end{figure}



\section{Вывод}

В этой работе мы познакомились с устройством лазера и интерферометра Фабри-Перо. А так же исследовали спектральные характеристики излучения.

\end{document}